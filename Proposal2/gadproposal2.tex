\documentclass[12pt]{article}
\usepackage{fullpage}
\usepackage{natbib}

\begin{document}
\author{Hannah Chu, Priyanka Kulshreshtha, Michael Gonzalez and Paula Quach}
\title{Gadgetron Project Proposal}
\maketitle

\section{Research Context and Problem Statement}
    What is the context for the work you are doing, why should people care about your work, and what is the specific problem you will solve?
    
    Gadgetron focuses on automating the process of building gadgets by allowing users a visual platform on which they 
	design their desired gadget. It then generates the appropriate code needed to be able to print out the circuit board.  
 	It has greatly reduced the limitations, making it possible to design a gadget without having to know much about
	the process. We would like to take this one step further and provide a way to automatically generate the appropriate
	code for the the function of the gadget. We believe that this a necessary step to acheive Gadgetron's overall goal. 			Adding this piece will reduce the constraints even further and enable people with very little technical knowledge 
	to make a working gadget. All a user would need to do is assemble it.\\[7pt]
	Perhaps the most successful model of this type is the LEGO Mindstorm model. It allows users to program their
	robot by using the accompanying Intelligence Brick. Although the kit claims to give users full freedom on how to 
	design their robot, they have a constraint in the form of the I-Brick [3]. No matter what designs they choose, that 
	block needs to be included. Gadgetron doesn't have any such constraint and truly gives the user the unique 
	experience of designing a robot according to their preferences.\\[7pt]
	There has also been a few attempts at creating something similar to our idea. One such system was the visual
	robot programming for generalizable mobile manipulation tasks project at the University of Washington. Their goal 
	was to make a product, called RoboFlow, that could replace tradditional coding when it came to programming
	robots[2]. The difference is usability. The format of Gadgetron' s Code Generation Framework is dedicated towards
	providing the easiest way to programm robots without having to know much about the process itself. This is 
	different from RoboFlow, in which you do need to have a basic understanding of harware engineering. Again, 
	Gadgetron would get rid of this limitation, making the tool more universal.

\subsection{Project Description}
	Our project describes the Code Generation Framework that was started by a graduate student Alexander Caughron.
	Bascally, we aim to extend Gadgetron to be able to automate the process of writing code for our gadget as well. 
	To do this, we plan to create a visual-based language that will enable users to outline what they want their gadget
	to do. The program will then generate the necessary code that is compatible with the gadget made using 
	Gadgetron's web tool.\\[7pt]
	We aim to make our tool more universal and cater to a larger audience. With our project added to Gadgetron, 
	it will become easy to use for any age group as it will require very little prior knowledge. We can introduce the 
	tool, and therefore the process of creating gadgets, at an early age which will motivate students to pursue it
	in the future. There have been trials using LEGO MindStorm to teach a group of junior high kids about robotics [1]. 
	With the levels of abstraction that Gadgetron provides, students can start learning about this stuff as early as 
	in elementary school. It can add to the exisiting cirriculum at lower level schools and include a displine to learn
	about other than the usual math, science, history, etc.\\[7pt]
	Gadgetron is already a novel project in that it attempts to automate such a complex process.  Combined
	with the Code Generation Framework, Gadgetron will become one of the easiest ways to create one's very one
	gadget. It will aide in teacihng kids about engineering from an early age.  

\section{Proposed Solution}
What is your specific proposed solution to the problem you identified in part 1?

\section{Evaluation and Implementation Plan}
How will you know if your research was successful?  And how will you organize your work to get it done in the time you have?  This section will have two sub-headers: "Evaluation Plan" and "Timeline"
Stuff here

\bibliographystyle{plain}
\bibliography{gadgetronprop}
\section {References}
[1] Practical Report of Programming Experiment Class for Elementary School Children
\newline
[2] Visual Robot Programming for Generalizable Mobile Manipulation Task 
\newline
[3]  Programming LEGO Mindstorms NXT with visual programming
\newline

\end{document}